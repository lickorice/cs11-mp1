%% Generated by Sphinx.
\def\sphinxdocclass{report}
\documentclass[letterpaper,10pt,english,openany,oneside]{sphinxmanual}
\ifdefined\pdfpxdimen
   \let\sphinxpxdimen\pdfpxdimen\else\newdimen\sphinxpxdimen
\fi \sphinxpxdimen=.75bp\relax

\PassOptionsToPackage{warn}{textcomp}
\usepackage[utf8]{inputenc}
\ifdefined\DeclareUnicodeCharacter
% support both utf8 and utf8x syntaxes
\edef\sphinxdqmaybe{\ifdefined\DeclareUnicodeCharacterAsOptional\string"\fi}
  \DeclareUnicodeCharacter{\sphinxdqmaybe00A0}{\nobreakspace}
  \DeclareUnicodeCharacter{\sphinxdqmaybe2500}{\sphinxunichar{2500}}
  \DeclareUnicodeCharacter{\sphinxdqmaybe2502}{\sphinxunichar{2502}}
  \DeclareUnicodeCharacter{\sphinxdqmaybe2514}{\sphinxunichar{2514}}
  \DeclareUnicodeCharacter{\sphinxdqmaybe251C}{\sphinxunichar{251C}}
  \DeclareUnicodeCharacter{\sphinxdqmaybe2572}{\textbackslash}
\fi
\usepackage{cmap}
\usepackage[T1]{fontenc}
\usepackage{amsmath,amssymb,amstext}
\usepackage{babel}
\usepackage{times}
\usepackage[Bjarne]{fncychap}
\usepackage{sphinx}

\fvset{fontsize=\small}
\usepackage{geometry}

% Include hyperref last.
\usepackage{hyperref}
% Fix anchor placement for figures with captions.
\usepackage{hypcap}% it must be loaded after hyperref.
% Set up styles of URL: it should be placed after hyperref.
\urlstyle{same}
\addto\captionsenglish{\renewcommand{\contentsname}{Contents:}}

\addto\captionsenglish{\renewcommand{\figurename}{Fig.}}
\addto\captionsenglish{\renewcommand{\tablename}{Table}}
\addto\captionsenglish{\renewcommand{\literalblockname}{Listing}}

\addto\captionsenglish{\renewcommand{\literalblockcontinuedname}{continued from previous page}}
\addto\captionsenglish{\renewcommand{\literalblockcontinuesname}{continues on next page}}
\addto\captionsenglish{\renewcommand{\sphinxnonalphabeticalgroupname}{Non-alphabetical}}
\addto\captionsenglish{\renewcommand{\sphinxsymbolsname}{Symbols}}
\addto\captionsenglish{\renewcommand{\sphinxnumbersname}{Numbers}}

\addto\extrasenglish{\def\pageautorefname{page}}

\setcounter{tocdepth}{1}



\title{Unscramble (CS11-MP1) Documentation}
\date{Nov 04, 2018}
\release{1.0}
\author{Carlos Panganiban, 2018-12497}
\newcommand{\sphinxlogo}{\vbox{}}
\renewcommand{\releasename}{Release}
\makeindex
\begin{document}

\pagestyle{empty}
\maketitle
\pagestyle{plain}
\sphinxtableofcontents
\pagestyle{normal}
\phantomsection\label{\detokenize{index::doc}}



\chapter{Installation}
\label{\detokenize{index:installation}}
The whole game can be copied to your directory
by downloading the source code \sphinxhref{http://www.github.com/lickorice/cs11-mp1}{here} or by cloning
the git repository like so:

\fvset{hllines={, ,}}%
\begin{sphinxVerbatim}[commandchars=\\\{\}]
git clone http://www.github.com/lickorice/cs11\PYGZhy{}mp1.git
\end{sphinxVerbatim}

Also, make sure you have \sphinxstylestrong{Pygame} installed for \sphinxstyleemphasis{Python 3}.
You can install Pygame through \sphinxcode{\sphinxupquote{pip}} like so:

\fvset{hllines={, ,}}%
\begin{sphinxVerbatim}[commandchars=\\\{\}]
pip3 install pygame
\end{sphinxVerbatim}

To start the game, run the game through Python 3 in the command terminal
with the following command (given that you are in the root directory of the
program):

\fvset{hllines={, ,}}%
\begin{sphinxVerbatim}[commandchars=\\\{\}]
py \PYGZhy{}3 main.py
\end{sphinxVerbatim}

In the absence of Pygame, \sphinxstylestrong{the CLI version of the game will boot instead}.


\chapter{Configuration}
\label{\detokenize{index:configuration}}

\section{Choosing a dictionary}
\label{\detokenize{index:choosing-a-dictionary}}
In the \sphinxcode{\sphinxupquote{config}} folder of the program, in \sphinxcode{\sphinxupquote{cfg\_general.json}}, you can change the directory of the
dictionary file to be used. Currently, the game has three built-in dictionaries, \sphinxcode{\sphinxupquote{dictionary\_sample.txt}},
\sphinxcode{\sphinxupquote{dictionary\_small.txt}}, and by default, \sphinxcode{\sphinxupquote{dictionary.txt}}. Simply change the parameters here:

\fvset{hllines={, ,}}%
\begin{sphinxVerbatim}[commandchars=\\\{\}]
\PYG{p}{\PYGZob{}}
   \PYG{n+nt}{\PYGZdq{}DICTIONARY\PYGZus{}FILE\PYGZdq{}}\PYG{p}{:} \PYG{l+s+s2}{\PYGZdq{}assets/dictionary.txt\PYGZdq{}}
\PYG{p}{\PYGZcb{}}
\end{sphinxVerbatim}


\chapter{How to Play}
\label{\detokenize{index:how-to-play}}
Upon starting the game, you will be welcomed by the
main menu screen. Here, you can choose from the two
game modes: \sphinxstylestrong{Anagram Mode} and \sphinxstylestrong{Combine Mode}. You may
also choose to \sphinxstylestrong{quit the application}; and you can quit the
application any time in the game by clicking on the default
close button of the program window.

\noindent{\hspace*{\fill}\sphinxincludegraphics{{1}.jpg}\hspace*{\fill}}

\newpage


\section{Anagram Mode}
\label{\detokenize{index:anagram-mode}}
In Anagram Mode, you are tasked to find anagrams of the word
shown on the screen. The word and its anagrams are based off
the dictionary file you used. To play, simply use your keyboard
to type, and pressing \sphinxtitleref{Enter} will confirm your answer.

\noindent{\hspace*{\fill}\sphinxincludegraphics{{2}.jpg}\hspace*{\fill}}


\section{Combine Mode}
\label{\detokenize{index:combine-mode}}
In Combine Mode, you are to find words that use letters from
the letter pool below the answer line. You may only use a letter
once in an answer, and you can’t repeat previously correct answers.

\noindent{\hspace*{\fill}\sphinxincludegraphics{{3}.jpg}\hspace*{\fill}}


\section{Ending the Game}
\label{\detokenize{index:ending-the-game}}
Both game modes will end when the \sphinxstylestrong{timer bar on the bottom is emptied}
(in 100 seconds) or when you \sphinxstylestrong{hit three mistakes}. Other than that,
it is possible to end the game prematurely when you have \sphinxstylestrong{found all possible anagrams of a word} (Anagram Mode)
or when you have \sphinxstylestrong{reached the absolute maximum number of points achievable} (Combine Mode).

In the score screen after the game ends, either your equivalent scrabble points or your number
of words solved shows up, in Combine Mode and Anagram Mode, respectively.

\noindent{\hspace*{\fill}\sphinxincludegraphics{{4}.jpg}\hspace*{\fill}}

\noindent{\hspace*{\fill}\sphinxincludegraphics{{5}.jpg}\hspace*{\fill}}


\chapter{Source Code Documentation}
\label{\detokenize{index:source-code-documentation}}

\section{\sphinxstyleliteralintitle{\sphinxupquote{main.py}}}
\label{\detokenize{index:module-main}}\label{\detokenize{index:main-py}}\index{main (module)}
This is the .py file for starting the game
Run this to run the game.
\index{main() (in module main)}

\begin{fulllineitems}
\phantomsection\label{\detokenize{index:main.main}}\pysiglinewithargsret{\sphinxcode{\sphinxupquote{main.}}\sphinxbfcode{\sphinxupquote{main}}}{}{}
The main method that is run when running the game.

\end{fulllineitems}



\section{\sphinxstyleliteralintitle{\sphinxupquote{interface.py}}}
\label{\detokenize{index:module-interface}}\label{\detokenize{index:interface-py}}\index{interface (module)}
This is the .py file for the interface
framework. (PyGame)
\index{add\_button() (in module interface)}

\begin{fulllineitems}
\phantomsection\label{\detokenize{index:interface.add_button}}\pysiglinewithargsret{\sphinxcode{\sphinxupquote{interface.}}\sphinxbfcode{\sphinxupquote{add\_button}}}{\emph{x1}, \emph{x2}, \emph{y1}, \emph{y2}, \emph{select\_state}, \emph{button\_action}, \emph{audio\_url=None}, \emph{pointer\_url=None}}{}
This function instantiates a button rectangle. It returns a Boolean whether or not the mouse
is hovering over it. The Boolean value is to be passed again into \sphinxcode{\sphinxupquote{select\_state}} in order to
avoid recursive effects (such as the hover audio playing again).
\begin{quote}\begin{description}
\item[{Returns}] \leavevmode
(\sphinxtitleref{bool}) \sphinxcode{\sphinxupquote{select\_state}}.

\item[{Parameters}] \leavevmode\begin{itemize}
\item {} 
\sphinxstyleliteralstrong{\sphinxupquote{x1}} (\sphinxstyleliteralemphasis{\sphinxupquote{int}}) \textendash{} coordinates on the x-axis where the rectangle starts.

\item {} 
\sphinxstyleliteralstrong{\sphinxupquote{x2}} (\sphinxstyleliteralemphasis{\sphinxupquote{int}}) \textendash{} coordinates on the x-axis where the rectangle ends.

\item {} 
\sphinxstyleliteralstrong{\sphinxupquote{y1}} (\sphinxstyleliteralemphasis{\sphinxupquote{int}}) \textendash{} coordinates on the y-axis where the rectangle starts.

\item {} 
\sphinxstyleliteralstrong{\sphinxupquote{y2}} (\sphinxstyleliteralemphasis{\sphinxupquote{int}}) \textendash{} coordinates on the y-axis where the rectangle ends.

\item {} 
\sphinxstyleliteralstrong{\sphinxupquote{select\_state}} (\sphinxstyleliteralemphasis{\sphinxupquote{bool}}) \textendash{} if the button is currently hovered on by the mouse.

\item {} 
\sphinxstyleliteralstrong{\sphinxupquote{button\_action}} (\sphinxstyleliteralemphasis{\sphinxupquote{function}}) \textendash{} the function to be performed on button click.

\item {} 
\sphinxstyleliteralstrong{\sphinxupquote{audio\_url}} (\sphinxstyleliteralemphasis{\sphinxupquote{string}}) \textendash{} path for the audio file to be played on hover.

\item {} 
\sphinxstyleliteralstrong{\sphinxupquote{pointer\_url}} (\sphinxstyleliteralemphasis{\sphinxupquote{string}}) \textendash{} path for the pointer image to be rendered on hover.

\end{itemize}

\end{description}\end{quote}

By default, \sphinxtitleref{audio\_url} and \sphinxtitleref{pointer\_url} are assigned \sphinxtitleref{None}, and will not play any audio
and show any pointers on button hover.

\end{fulllineitems}

\index{anagram\_loading\_screen() (in module interface)}

\begin{fulllineitems}
\phantomsection\label{\detokenize{index:interface.anagram_loading_screen}}\pysiglinewithargsret{\sphinxcode{\sphinxupquote{interface.}}\sphinxbfcode{\sphinxupquote{anagram\_loading\_screen}}}{}{}
This function instantiates the anagram game loading screen.

This is called by \sphinxtitleref{interface.anagram\_screen} and \sphinxstylestrong{should not be called directly}.

\end{fulllineitems}

\index{anagram\_score\_screen() (in module interface)}

\begin{fulllineitems}
\phantomsection\label{\detokenize{index:interface.anagram_score_screen}}\pysiglinewithargsret{\sphinxcode{\sphinxupquote{interface.}}\sphinxbfcode{\sphinxupquote{anagram\_score\_screen}}}{\emph{answer\_list}}{}
This function instantiates the anagram score screen instance.

This is called by \sphinxtitleref{interface.anagram\_screen} and \sphinxstylestrong{should not be called directly}.
\begin{quote}\begin{description}
\item[{Parameters}] \leavevmode
\sphinxstyleliteralstrong{\sphinxupquote{answer\_list}} (\sphinxstyleliteralemphasis{\sphinxupquote{list}}) \textendash{} a list of words the user has correctly answered.

\end{description}\end{quote}

\end{fulllineitems}

\index{anagram\_screen() (in module interface)}

\begin{fulllineitems}
\phantomsection\label{\detokenize{index:interface.anagram_screen}}\pysiglinewithargsret{\sphinxcode{\sphinxupquote{interface.}}\sphinxbfcode{\sphinxupquote{anagram\_screen}}}{}{}
This function instantiates the anagram game instance.

This function is called directly from \sphinxtitleref{interface.start\_menu} through a button click.

\end{fulllineitems}

\index{back\_button() (in module interface)}

\begin{fulllineitems}
\phantomsection\label{\detokenize{index:interface.back_button}}\pysiglinewithargsret{\sphinxcode{\sphinxupquote{interface.}}\sphinxbfcode{\sphinxupquote{back\_button}}}{\emph{x1}, \emph{x2}, \emph{y1}, \emph{y2}, \emph{select\_state}, \emph{button\_action}, \emph{audio\_url=None}}{}
This function instantiates a back button rectangle.

Functionality is the same with \sphinxcode{\sphinxupquote{interface.add\_button}}.

\end{fulllineitems}

\index{combine\_score\_screen() (in module interface)}

\begin{fulllineitems}
\phantomsection\label{\detokenize{index:interface.combine_score_screen}}\pysiglinewithargsret{\sphinxcode{\sphinxupquote{interface.}}\sphinxbfcode{\sphinxupquote{combine\_score\_screen}}}{\emph{letter\_string}, \emph{max\_points}}{}
This function instantiates the combine score screen instance.

This is called by \sphinxtitleref{interface.combine\_screen} and \sphinxstylestrong{should not be called directly}.
\begin{quote}\begin{description}
\item[{Parameters}] \leavevmode\begin{itemize}
\item {} 
\sphinxstyleliteralstrong{\sphinxupquote{letter\_string}} (\sphinxstyleliteralemphasis{\sphinxupquote{string}}) \textendash{} the randomly generated string used in the game mode.

\item {} 
\sphinxstyleliteralstrong{\sphinxupquote{max\_points}} (\sphinxstyleliteralemphasis{\sphinxupquote{int}}) \textendash{} the maximum number of points achievable with the string.

\end{itemize}

\end{description}\end{quote}

\end{fulllineitems}

\index{combine\_screen() (in module interface)}

\begin{fulllineitems}
\phantomsection\label{\detokenize{index:interface.combine_screen}}\pysiglinewithargsret{\sphinxcode{\sphinxupquote{interface.}}\sphinxbfcode{\sphinxupquote{combine\_screen}}}{}{}
This function instantiates the combine game instance.

This function is called directly from \sphinxtitleref{interface.start\_menu} through a button click.

\end{fulllineitems}

\index{fade() (in module interface)}

\begin{fulllineitems}
\phantomsection\label{\detokenize{index:interface.fade}}\pysiglinewithargsret{\sphinxcode{\sphinxupquote{interface.}}\sphinxbfcode{\sphinxupquote{fade}}}{\emph{background\_url}, \emph{fade\_type='out'}, \emph{time\_delay=3}}{}
This function fades a background in or out, given its url and fade type.
\begin{quote}\begin{description}
\item[{Parameters}] \leavevmode\begin{itemize}
\item {} 
\sphinxstyleliteralstrong{\sphinxupquote{background\_url}} (\sphinxstyleliteralemphasis{\sphinxupquote{string}}) \textendash{} path for background image.

\item {} 
\sphinxstyleliteralstrong{\sphinxupquote{fade\_type}} (\sphinxstyleliteralemphasis{\sphinxupquote{string}}) \textendash{} {[}\sphinxcode{\sphinxupquote{"out"}} /\sphinxcode{\sphinxupquote{"in"}}{]} for fade out or fade in, respectively.

\item {} 
\sphinxstyleliteralstrong{\sphinxupquote{time\_delay}} (\sphinxstyleliteralemphasis{\sphinxupquote{int}}) \textendash{} delay in seconds before fading.

\end{itemize}

\end{description}\end{quote}

\end{fulllineitems}

\index{mistakes() (in module interface)}

\begin{fulllineitems}
\phantomsection\label{\detokenize{index:interface.mistakes}}\pysiglinewithargsret{\sphinxcode{\sphinxupquote{interface.}}\sphinxbfcode{\sphinxupquote{mistakes}}}{\emph{count}}{}
This function instantiates a mistakes counter on the screen. By default, this renders for a maximum of three
mistakes per game. The boolean returned by this function indicates if the user has already made three mistakes.
\begin{quote}\begin{description}
\item[{Returns}] \leavevmode
(\sphinxtitleref{bool}) \sphinxtitleref{True} if mistakes reach three, \sphinxtitleref{False} if otherwise.

\item[{Parameters}] \leavevmode
\sphinxstyleliteralstrong{\sphinxupquote{count}} (\sphinxstyleliteralemphasis{\sphinxupquote{int}}) \textendash{} number of mistakes by the player

\end{description}\end{quote}

\end{fulllineitems}

\index{splash\_screen() (in module interface)}

\begin{fulllineitems}
\phantomsection\label{\detokenize{index:interface.splash_screen}}\pysiglinewithargsret{\sphinxcode{\sphinxupquote{interface.}}\sphinxbfcode{\sphinxupquote{splash\_screen}}}{}{}
This function instantiates the splash screen.

\end{fulllineitems}

\index{start\_game() (in module interface)}

\begin{fulllineitems}
\phantomsection\label{\detokenize{index:interface.start_game}}\pysiglinewithargsret{\sphinxcode{\sphinxupquote{interface.}}\sphinxbfcode{\sphinxupquote{start\_game}}}{}{}
This function starts the whole game.

\end{fulllineitems}

\index{start\_menu() (in module interface)}

\begin{fulllineitems}
\phantomsection\label{\detokenize{index:interface.start_menu}}\pysiglinewithargsret{\sphinxcode{\sphinxupquote{interface.}}\sphinxbfcode{\sphinxupquote{start\_menu}}}{}{}
This function instantiates the game menu.

This is called when a game ends or when the splash screen ends 
and \sphinxstylestrong{should not be called directly}.

\end{fulllineitems}

\index{start\_transition() (in module interface)}

\begin{fulllineitems}
\phantomsection\label{\detokenize{index:interface.start_transition}}\pysiglinewithargsret{\sphinxcode{\sphinxupquote{interface.}}\sphinxbfcode{\sphinxupquote{start\_transition}}}{}{}
This function acts as a bridging instance for the menu.

This is called when a game ends and \sphinxstylestrong{should not be called directly}.

\end{fulllineitems}

\index{swipe() (in module interface)}

\begin{fulllineitems}
\phantomsection\label{\detokenize{index:interface.swipe}}\pysiglinewithargsret{\sphinxcode{\sphinxupquote{interface.}}\sphinxbfcode{\sphinxupquote{swipe}}}{\emph{background\_url}}{}
This function sweeps a background in from the bottom, given its url.
\begin{quote}\begin{description}
\item[{Parameters}] \leavevmode
\sphinxstyleliteralstrong{\sphinxupquote{background\_url}} (\sphinxstyleliteralemphasis{\sphinxupquote{string}}) \textendash{} path for background image.

\end{description}\end{quote}

\end{fulllineitems}

\index{text\_blit() (in module interface)}

\begin{fulllineitems}
\phantomsection\label{\detokenize{index:interface.text_blit}}\pysiglinewithargsret{\sphinxcode{\sphinxupquote{interface.}}\sphinxbfcode{\sphinxupquote{text\_blit}}}{\emph{text}, \emph{font\_size}, \emph{font\_url}, \emph{rgb}, \emph{center=True}, \emph{x=None}, \emph{y=None}}{}
This renders given text onto the game screen.
\begin{quote}\begin{description}
\item[{Parameters}] \leavevmode\begin{itemize}
\item {} 
\sphinxstyleliteralstrong{\sphinxupquote{text}} (\sphinxstyleliteralemphasis{\sphinxupquote{string}}) \textendash{} literal string of text to be rendered.

\item {} 
\sphinxstyleliteralstrong{\sphinxupquote{font\_size}} (\sphinxstyleliteralemphasis{\sphinxupquote{int}}) \textendash{} font size (in pixels) of text.

\item {} 
\sphinxstyleliteralstrong{\sphinxupquote{font\_url}} (\sphinxstyleliteralemphasis{\sphinxupquote{string}}) \textendash{} path for the font file \sphinxtitleref{(.ttf/.otf)} of the font.

\item {} 
\sphinxstyleliteralstrong{\sphinxupquote{rgb}} (\sphinxstyleliteralemphasis{\sphinxupquote{list/tuple}}) \textendash{} {[}\sphinxcode{\sphinxupquote{red}}, \sphinxcode{\sphinxupquote{green}}, \sphinxcode{\sphinxupquote{blue}}{]} int tuple/list for the RGB color.

\item {} 
\sphinxstyleliteralstrong{\sphinxupquote{center}} (\sphinxstyleliteralemphasis{\sphinxupquote{bool}}) \textendash{} if coordinates are aligned according to text’s geometric center.

\item {} 
\sphinxstyleliteralstrong{\sphinxupquote{x}} (\sphinxstyleliteralemphasis{\sphinxupquote{int}}) \textendash{} coordinates of render location on the x-axis.

\item {} 
\sphinxstyleliteralstrong{\sphinxupquote{y}} (\sphinxstyleliteralemphasis{\sphinxupquote{int}}) \textendash{} coordinates of render location on the y-axis.

\end{itemize}

\end{description}\end{quote}

If \sphinxtitleref{x} and \sphinxtitleref{y} are both left \sphinxtitleref{None}, the text is automatically rendered on the center of the screen.

If \sphinxtitleref{center} is on \sphinxtitleref{False}, the text is rendered with \sphinxtitleref{x} and \sphinxtitleref{y} as its top-right orientation.

\end{fulllineitems}

\index{timer() (in module interface)}

\begin{fulllineitems}
\phantomsection\label{\detokenize{index:interface.timer}}\pysiglinewithargsret{\sphinxcode{\sphinxupquote{interface.}}\sphinxbfcode{\sphinxupquote{timer}}}{\emph{count}}{}
This function instantiates a timer on the screen. By default, the timer runs for \sphinxstylestrong{100 seconds}.
\begin{quote}\begin{description}
\item[{Returns}] \leavevmode
(\sphinxtitleref{bool}) \sphinxtitleref{True} if timer has ended and \sphinxtitleref{False} if otherwise.

\item[{Parameters}] \leavevmode
\sphinxstyleliteralstrong{\sphinxupquote{count}} (\sphinxstyleliteralemphasis{\sphinxupquote{float}}) \textendash{} current count in seconds.

\end{description}\end{quote}

\end{fulllineitems}



\section{\sphinxstyleliteralintitle{\sphinxupquote{cli\_interface.py}}}
\label{\detokenize{index:module-cli_interface}}\label{\detokenize{index:cli-interface-py}}\index{cli\_interface (module)}
This module is only run when \sphinxstylestrong{Pygame is not installed in the system.}
\index{anagram\_screen() (in module cli\_interface)}

\begin{fulllineitems}
\phantomsection\label{\detokenize{index:cli_interface.anagram_screen}}\pysiglinewithargsret{\sphinxcode{\sphinxupquote{cli\_interface.}}\sphinxbfcode{\sphinxupquote{anagram\_screen}}}{}{}
This function instantiates the anagram screen.

\end{fulllineitems}

\index{combine\_screen() (in module cli\_interface)}

\begin{fulllineitems}
\phantomsection\label{\detokenize{index:cli_interface.combine_screen}}\pysiglinewithargsret{\sphinxcode{\sphinxupquote{cli\_interface.}}\sphinxbfcode{\sphinxupquote{combine\_screen}}}{}{}
This function instantiates the anagram screen.

\end{fulllineitems}

\index{input\_prompt() (in module cli\_interface)}

\begin{fulllineitems}
\phantomsection\label{\detokenize{index:cli_interface.input_prompt}}\pysiglinewithargsret{\sphinxcode{\sphinxupquote{cli\_interface.}}\sphinxbfcode{\sphinxupquote{input\_prompt}}}{\emph{mistakes}}{}
This is the equivalent of the mistake counter
in the original Pygame rendition. However,
this is integrated to the input prompt in the CLI.
\begin{quote}\begin{description}
\item[{Returns}] \leavevmode
(\sphinxtitleref{bool}) if the time has ended or if mistakes reach 3, (\sphinxtitleref{string}) the prompt string.

\item[{Parameters}] \leavevmode
\sphinxstyleliteralstrong{\sphinxupquote{mistakes}} (\sphinxstyleliteralemphasis{\sphinxupquote{int}}) \textendash{} the number of mistakes the player currently has.

\end{description}\end{quote}

\end{fulllineitems}

\index{start\_game() (in module cli\_interface)}

\begin{fulllineitems}
\phantomsection\label{\detokenize{index:cli_interface.start_game}}\pysiglinewithargsret{\sphinxcode{\sphinxupquote{cli\_interface.}}\sphinxbfcode{\sphinxupquote{start\_game}}}{}{}
This function starts the game in CLI mode.

\end{fulllineitems}

\index{start\_menu() (in module cli\_interface)}

\begin{fulllineitems}
\phantomsection\label{\detokenize{index:cli_interface.start_menu}}\pysiglinewithargsret{\sphinxcode{\sphinxupquote{cli\_interface.}}\sphinxbfcode{\sphinxupquote{start\_menu}}}{}{}
This function shows the main menu in CLI mode.

\end{fulllineitems}



\section{\sphinxstyleliteralintitle{\sphinxupquote{engine.py}}}
\label{\detokenize{index:module-engine}}\label{\detokenize{index:engine-py}}\index{engine (module)}
This is the .py file for the game engine.
\index{anagram\_correct() (in module engine)}

\begin{fulllineitems}
\phantomsection\label{\detokenize{index:engine.anagram_correct}}\pysiglinewithargsret{\sphinxcode{\sphinxupquote{engine.}}\sphinxbfcode{\sphinxupquote{anagram\_correct}}}{}{}
This function increments the number of words solved by the player.

\end{fulllineitems}

\index{anagram\_end() (in module engine)}

\begin{fulllineitems}
\phantomsection\label{\detokenize{index:engine.anagram_end}}\pysiglinewithargsret{\sphinxcode{\sphinxupquote{engine.}}\sphinxbfcode{\sphinxupquote{anagram\_end}}}{}{}
This function returns the number of words solved by the player.
\begin{quote}\begin{description}
\item[{Returns}] \leavevmode
(\sphinxtitleref{int}) player’s scrabble points upon game end.

\end{description}\end{quote}

\end{fulllineitems}

\index{anagram\_init() (in module engine)}

\begin{fulllineitems}
\phantomsection\label{\detokenize{index:engine.anagram_init}}\pysiglinewithargsret{\sphinxcode{\sphinxupquote{engine.}}\sphinxbfcode{\sphinxupquote{anagram\_init}}}{}{}
This function starts the anagram game.

In addition, it also returns a word (regardless of the number of anagrams)
for the loading sequence to show.
\begin{quote}\begin{description}
\item[{Returns}] \leavevmode
(\sphinxtitleref{string}) temporary target word for the game, (\sphinxtitleref{list}) list of anagrams.

\end{description}\end{quote}

\end{fulllineitems}

\index{combine\_correct() (in module engine)}

\begin{fulllineitems}
\phantomsection\label{\detokenize{index:engine.combine_correct}}\pysiglinewithargsret{\sphinxcode{\sphinxupquote{engine.}}\sphinxbfcode{\sphinxupquote{combine\_correct}}}{\emph{word}, \emph{letter\_string}}{}
This function checks whether or not the player got the answer correctly.

If the player is correct, it increments the players points, and returns
a \sphinxtitleref{True} value. Otherwise, it does nothing and returns a \sphinxtitleref{False} value.
\begin{quote}\begin{description}
\item[{Returns}] \leavevmode
(\sphinxtitleref{bool}) check result.

\end{description}\end{quote}

\end{fulllineitems}

\index{combine\_end() (in module engine)}

\begin{fulllineitems}
\phantomsection\label{\detokenize{index:engine.combine_end}}\pysiglinewithargsret{\sphinxcode{\sphinxupquote{engine.}}\sphinxbfcode{\sphinxupquote{combine\_end}}}{}{}
This function returns the total points achieved by the player
upon game end.
\begin{quote}\begin{description}
\item[{Returns}] \leavevmode
(\sphinxtitleref{int}) player points.

\end{description}\end{quote}

\end{fulllineitems}

\index{combine\_init() (in module engine)}

\begin{fulllineitems}
\phantomsection\label{\detokenize{index:engine.combine_init}}\pysiglinewithargsret{\sphinxcode{\sphinxupquote{engine.}}\sphinxbfcode{\sphinxupquote{combine\_init}}}{}{}
This function starts the combine game.

In addition, it returns the string and the maximum points achievable
using the string during the game.
\begin{quote}\begin{description}
\item[{Returns}] \leavevmode
(\sphinxtitleref{string}) letter pool, (\sphinxtitleref{int}) maximum points.

\end{description}\end{quote}

\end{fulllineitems}

\index{combine\_points() (in module engine)}

\begin{fulllineitems}
\phantomsection\label{\detokenize{index:engine.combine_points}}\pysiglinewithargsret{\sphinxcode{\sphinxupquote{engine.}}\sphinxbfcode{\sphinxupquote{combine\_points}}}{}{}
This function returns the total points of the player.
\begin{quote}\begin{description}
\item[{Returns}] \leavevmode
(\sphinxtitleref{int}) player points.

\end{description}\end{quote}

\end{fulllineitems}

\index{init\_dictionary() (in module engine)}

\begin{fulllineitems}
\phantomsection\label{\detokenize{index:engine.init_dictionary}}\pysiglinewithargsret{\sphinxcode{\sphinxupquote{engine.}}\sphinxbfcode{\sphinxupquote{init\_dictionary}}}{\emph{filename}}{}
This function initializes the game’s dictionary given a filename.
\begin{quote}\begin{description}
\item[{Returns}] \leavevmode
(\sphinxtitleref{bool}) if the dictionary has been initialized correctly.

\item[{Parameters}] \leavevmode
\sphinxstyleliteralstrong{\sphinxupquote{filename}} (\sphinxstyleliteralemphasis{\sphinxupquote{string}}) \textendash{} path for the dictionary (this is set in the config file).

\end{description}\end{quote}

\end{fulllineitems}



\section{\sphinxstyleliteralintitle{\sphinxupquote{anagram.py}}}
\label{\detokenize{index:module-anagram}}\label{\detokenize{index:anagram-py}}\index{anagram (module)}
This contains the code and logic for the
first gamemode, SEARCHING FOR ANAGRAMS.

Functions in this file are used by \sphinxtitleref{engine.py}
and are not meant to be used directly by neither
\sphinxtitleref{interface.py} nor \sphinxtitleref{main.py}.
\index{anagrams() (in module anagram)}

\begin{fulllineitems}
\phantomsection\label{\detokenize{index:anagram.anagrams}}\pysiglinewithargsret{\sphinxcode{\sphinxupquote{anagram.}}\sphinxbfcode{\sphinxupquote{anagrams}}}{\emph{target\_word}, \emph{input\_dict}}{}
This function returns a list of words given an anagram and a dictionary.
\begin{quote}\begin{description}
\item[{Returns}] \leavevmode
(\sphinxtitleref{list}) list of anagrams.

\item[{Parameters}] \leavevmode\begin{itemize}
\item {} 
\sphinxstyleliteralstrong{\sphinxupquote{target\_word}} (\sphinxstyleliteralemphasis{\sphinxupquote{string}}) \textendash{} word used to search for anagrams.

\item {} 
\sphinxstyleliteralstrong{\sphinxupquote{input\_dict}} (\sphinxstyleliteralemphasis{\sphinxupquote{list}}) \textendash{} dictionary to be used to search for anagrams.

\end{itemize}

\end{description}\end{quote}

\end{fulllineitems}

\index{init\_word() (in module anagram)}

\begin{fulllineitems}
\phantomsection\label{\detokenize{index:anagram.init_word}}\pysiglinewithargsret{\sphinxcode{\sphinxupquote{anagram.}}\sphinxbfcode{\sphinxupquote{init\_word}}}{\emph{dictionary}, \emph{word\_count}}{}
This method instantiates a word with its corresponding anagram list.
\begin{quote}\begin{description}
\item[{Returns}] \leavevmode
(\sphinxtitleref{string}) random word, (\sphinxtitleref{list}) anagram list.

\item[{Parameters}] \leavevmode\begin{itemize}
\item {} 
\sphinxstyleliteralstrong{\sphinxupquote{dictionary}} (\sphinxstyleliteralemphasis{\sphinxupquote{list}}) \textendash{} dictionary list to be used.

\item {} 
\sphinxstyleliteralstrong{\sphinxupquote{word\_count}} (\sphinxstyleliteralemphasis{\sphinxupquote{int}}) \textendash{} total number of words in the dictionary.

\end{itemize}

\end{description}\end{quote}

\end{fulllineitems}



\section{\sphinxstyleliteralintitle{\sphinxupquote{combine.py}}}
\label{\detokenize{index:module-combine}}\label{\detokenize{index:combine-py}}\index{combine (module)}
This contains the code and logic for the
second gamemode, COMBINING WORDS.

Functions in this file are used by \sphinxtitleref{engine.py}
and are not meant to be used directly by neither
\sphinxtitleref{interface.py} nor \sphinxtitleref{main.py}.
\index{check\_answer() (in module combine)}

\begin{fulllineitems}
\phantomsection\label{\detokenize{index:combine.check_answer}}\pysiglinewithargsret{\sphinxcode{\sphinxupquote{combine.}}\sphinxbfcode{\sphinxupquote{check\_answer}}}{\emph{sequence\_str}, \emph{input\_str}, \emph{input\_dict}}{}
This function checks if your answer is within bounds of the
generated string and is valid based on the dictionary.
\begin{quote}\begin{description}
\item[{Returns}] \leavevmode
(\sphinxtitleref{bool}) \sphinxtitleref{True} or \sphinxtitleref{False} if answer is correct.

\item[{Parameters}] \leavevmode\begin{itemize}
\item {} 
\sphinxstyleliteralstrong{\sphinxupquote{sequence\_str}} (\sphinxstyleliteralemphasis{\sphinxupquote{string}}) \textendash{} the sequence string provided by the game.

\item {} 
\sphinxstyleliteralstrong{\sphinxupquote{input\_str}} (\sphinxstyleliteralemphasis{\sphinxupquote{string}}) \textendash{} the answer from the player.

\item {} 
\sphinxstyleliteralstrong{\sphinxupquote{input\_dict}} (\sphinxstyleliteralemphasis{\sphinxupquote{list}}) \textendash{} the dictionary to be used.

\end{itemize}

\end{description}\end{quote}

\end{fulllineitems}

\index{convert\_points() (in module combine)}

\begin{fulllineitems}
\phantomsection\label{\detokenize{index:combine.convert_points}}\pysiglinewithargsret{\sphinxcode{\sphinxupquote{combine.}}\sphinxbfcode{\sphinxupquote{convert\_points}}}{\emph{word}}{}
This converts the scrabble points of a word.
\begin{quote}\begin{description}
\item[{Returns}] \leavevmode
(\sphinxtitleref{int}) scrabble points.

\item[{Parameters}] \leavevmode
\sphinxstyleliteralstrong{\sphinxupquote{word}} (\sphinxstyleliteralemphasis{\sphinxupquote{string}}) \textendash{} input word.

\end{description}\end{quote}

\end{fulllineitems}

\index{generate\_sequence() (in module combine)}

\begin{fulllineitems}
\phantomsection\label{\detokenize{index:combine.generate_sequence}}\pysiglinewithargsret{\sphinxcode{\sphinxupquote{combine.}}\sphinxbfcode{\sphinxupquote{generate\_sequence}}}{\emph{word\_list}}{}
This function generates an absolute minimum dictionary of letters required
to form a given list of words.
\begin{quote}\begin{description}
\item[{Returns}] \leavevmode
(\sphinxtitleref{dict}) dictionary of letters and letter count.

\item[{Parameters}] \leavevmode
\sphinxstyleliteralstrong{\sphinxupquote{word\_list}} (\sphinxstyleliteralemphasis{\sphinxupquote{list}}) \textendash{} list of words to be used to generate the letters.

\end{description}\end{quote}

\end{fulllineitems}

\index{generate\_string() (in module combine)}

\begin{fulllineitems}
\phantomsection\label{\detokenize{index:combine.generate_string}}\pysiglinewithargsret{\sphinxcode{\sphinxupquote{combine.}}\sphinxbfcode{\sphinxupquote{generate\_string}}}{\emph{sequence}}{}
Given a sequence (alphabet dictionary), generate a string of \sphinxstyleemphasis{ordered} letters.
\begin{quote}\begin{description}
\item[{Returns}] \leavevmode
(\sphinxtitleref{string}) a randomly generated string made from the absolute minimum letter dictionary.

\item[{Parameters}] \leavevmode
\sphinxstyleliteralstrong{\sphinxupquote{sequence}} (\sphinxstyleliteralemphasis{\sphinxupquote{dict}}) \textendash{} an absolute minimum letter dictionary produced by \sphinxcode{\sphinxupquote{combine.generate\_sequence}}.

\end{description}\end{quote}

\end{fulllineitems}

\index{init\_letters() (in module combine)}

\begin{fulllineitems}
\phantomsection\label{\detokenize{index:combine.init_letters}}\pysiglinewithargsret{\sphinxcode{\sphinxupquote{combine.}}\sphinxbfcode{\sphinxupquote{init\_letters}}}{\emph{dictionary}, \emph{word\_count}}{}
This function generates a list of 16 letters that have valid 
answers based on the dictionary.
\begin{quote}\begin{description}
\item[{Returns}] \leavevmode
(\sphinxtitleref{string}) string of 16 letters, (\sphinxtitleref{int}) maximum points.

\item[{Parameters}] \leavevmode\begin{itemize}
\item {} 
\sphinxstyleliteralstrong{\sphinxupquote{dictionary}} (\sphinxstyleliteralemphasis{\sphinxupquote{list}}) \textendash{} dictionary to be used to generate the letters.

\item {} 
\sphinxstyleliteralstrong{\sphinxupquote{word\_count}} (\sphinxstyleliteralemphasis{\sphinxupquote{int}}) \textendash{} word count of the dictionary.

\end{itemize}

\end{description}\end{quote}

\end{fulllineitems}

\index{max\_points() (in module combine)}

\begin{fulllineitems}
\phantomsection\label{\detokenize{index:combine.max_points}}\pysiglinewithargsret{\sphinxcode{\sphinxupquote{combine.}}\sphinxbfcode{\sphinxupquote{max\_points}}}{\emph{input\_str}, \emph{input\_dict}}{}
This returns the maximum points (integer) that you can achieve
given a scrambled string (input\_str) and a dictionary (input\_dict).
\begin{quote}\begin{description}
\item[{Returns}] \leavevmode
(\sphinxtitleref{int}) maximum achievable points of the word.

\item[{Parameters}] \leavevmode\begin{itemize}
\item {} 
\sphinxstyleliteralstrong{\sphinxupquote{input\_str}} (\sphinxstyleliteralemphasis{\sphinxupquote{string}}) \textendash{} the generated letter pool string used in the game.

\item {} 
\sphinxstyleliteralstrong{\sphinxupquote{input\_dict}} (\sphinxstyleliteralemphasis{\sphinxupquote{list}}) \textendash{} the dictionary to be used.

\end{itemize}

\end{description}\end{quote}

\end{fulllineitems}



\section{\sphinxstyleliteralintitle{\sphinxupquote{player.py}}}
\label{\detokenize{index:module-player}}\label{\detokenize{index:player-py}}\index{player (module)}
This contains the Player class.
\index{Player (class in player)}

\begin{fulllineitems}
\phantomsection\label{\detokenize{index:player.Player}}\pysigline{\sphinxbfcode{\sphinxupquote{class }}\sphinxcode{\sphinxupquote{player.}}\sphinxbfcode{\sphinxupquote{Player}}}
A Player class to store points, words solved, and name.

This is for multiple purposes, such as word count recording,
point tallying, and etc.
\begin{quote}\begin{description}
\item[{Property}] \leavevmode
points

\item[{Type}] \leavevmode
int

\item[{Returns}] \leavevmode
The number of scrabble points the player has (Combine mode).

\item[{Property}] \leavevmode
words\_solved

\item[{Type}] \leavevmode
int

\item[{Returns}] \leavevmode
The number of words the player has solved (Anagram mode).

\end{description}\end{quote}

\end{fulllineitems}



\renewcommand{\indexname}{Python Module Index}
\begin{sphinxtheindex}
\let\bigletter\sphinxstyleindexlettergroup
\bigletter{a}
\item\relax\sphinxstyleindexentry{anagram}\sphinxstyleindexpageref{index:\detokenize{module-anagram}}
\indexspace
\bigletter{c}
\item\relax\sphinxstyleindexentry{cli\_interface}\sphinxstyleindexpageref{index:\detokenize{module-cli_interface}}
\item\relax\sphinxstyleindexentry{combine}\sphinxstyleindexpageref{index:\detokenize{module-combine}}
\indexspace
\bigletter{e}
\item\relax\sphinxstyleindexentry{engine}\sphinxstyleindexpageref{index:\detokenize{module-engine}}
\indexspace
\bigletter{i}
\item\relax\sphinxstyleindexentry{interface}\sphinxstyleindexpageref{index:\detokenize{module-interface}}
\indexspace
\bigletter{m}
\item\relax\sphinxstyleindexentry{main}\sphinxstyleindexpageref{index:\detokenize{module-main}}
\indexspace
\bigletter{p}
\item\relax\sphinxstyleindexentry{player}\sphinxstyleindexpageref{index:\detokenize{module-player}}
\end{sphinxtheindex}

\renewcommand{\indexname}{Index}
\printindex
\end{document}